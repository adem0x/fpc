\chapter{TQueue}

Implements queue.

Usage example:

\lstinputlisting[language=Pascal]{queueexample.pp}

Memory complexity:
Since underlaying structure is TDeque, memory complexity is same.

Members list:

\begin{longtable}{|m{10cm}|m{5cm}|}
\hline
Method & Complexity guarantees \\ \hline
\multicolumn{2}{|m{15cm}|}{Description} \\ \hline\hline

\verb!Create! & O(1) \\ \hline
\multicolumn{2}{|m{15cm}|}{Constructor. Creates empty queue.} \\ \hline\hline

\verb!function Size(): SizeUInt! & O(1) \\ \hline
\multicolumn{2}{|m{15cm}|}{Returns number of elements in queue.} \\\hline\hline

\verb!procedure Push(value: T)! &  Amortized
O(1), some operations might take O(N) time, when array needs to be reallocated, but sequence of N
operations takes O(N) time \\ \hline
\multicolumn{2}{|m{15cm}|}{Inserts element at the back of queue.} \\\hline\hline

\verb!procedure Pop()! & O(1) \\\hline
\multicolumn{2}{|m{15cm}|}{Removes element from the beginning of queue. If queue is empty does nothing.} \\\hline\hline

\verb!function IsEmpty(): boolean! & O(1) \\ \hline
\multicolumn{2}{|m{15cm}|}{Returns true when queue is empty.} \\\hline\hline

\verb!function Front: T! & O(1) \\\hline
\multicolumn{2}{|m{15cm}|}{Returns the first element from queue.} \\\hline

\end{longtable}
