\chapter{TStack}

Implements stack.

Usage example:

\lstinputlisting[language=Pascal]{stackexample.pp}

Memory complexity:
Since underlaying structure is TVector, memory complexity is same.

Members list:

\begin{longtable}{|m{10cm}|m{5cm}|}
\hline
Method & Complexity guarantees \\ \hline
\multicolumn{2}{|m{15cm}|}{Description} \\ \hline\hline

\verb!Create! & O(1) \\ \hline
\multicolumn{2}{|m{15cm}|}{Constructor. Creates empty stack.} \\ \hline\hline

\verb!function Size(): SizeUInt! & O(1) \\ \hline
\multicolumn{2}{|m{15cm}|}{Returns number of elements in stack.} \\\hline\hline

\verb!procedure Push(value: T)! &  Amortized
O(1), some operations might take O(N) time, when array needs to be reallocated, but sequence of N
operations takes O(N) time \\ \hline
\multicolumn{2}{|m{15cm}|}{Inserts element on the top of stack.} \\\hline\hline

\verb!procedure Pop()! & O(1) \\\hline
\multicolumn{2}{|m{15cm}|}{Removes element from the top of stack. If stack is empty does nothing.} \\\hline\hline

\verb!function IsEmpty(): boolean! & O(1) \\ \hline
\multicolumn{2}{|m{15cm}|}{Returns true when stack is empty} \\\hline\hline

\verb!function Top: T! & O(1) \\\hline
\multicolumn{2}{|m{15cm}|}{Returns top element from stack.} \\\hline

\end{longtable}
